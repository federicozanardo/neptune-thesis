%!TEX TS-program = pdflatex
%!TEX root = tesi.tex
%!TEX encoding = UTF-8 Unicode


       %%%%%%%%%%%%%%%%%%%%%%
       %                    %
       %  Introduzione.tex  %
       %                    %
       %%%%%%%%%%%%%%%%%%%%%%

\chapter{Introduzione}
Il lavoro descritto in questa tesi nasce dall'esperienza di tirocinio interno svolto presso il Dipartimento di Scienze Matematiche, Informatiche e Fisiche dell’Università degli Studi di Udine. L'obiettivo principale di questo tirocinio è stato quello di realizzare un'\textbf{applicazione mobile cross-platform} per il monitoraggio dei dati provenienti da un server installato su una barca a vela, in tempo reale. Il contesto d'uso di questa applicazione è la \textbf{regata}. Un altro obiettivo molto importante del tirocinio è stato quello di approfondire il mondo dello sviluppo di applicazioni cross-platform, valutandone pregi e difetti concretamente durante la  realizzazione dell'applicazione.

L'applicazione va ad integrare un progetto già avviato dall'Università degli Studi di Udine: il progetto \textbf{UniUD Sailing Lab}. È un progetto di ricerca del Dipartimento Politecnico di Ingegneria e Architettura dell’Università.

In passato sono state sviluppate sia un'applicazione Web che una per dispositivi Android per il medesimo scopo. L'applicazione Android riuscì a sopperire alcuni svantaggi della versione Web, aumentando l'usabilità dell'applicazione, migliorando la fruizione delle informazioni e migliorando le prestazioni complessive dell'app. Inoltre permetteva di utilizzare le caratteristiche intrinseche presenti nel dispositivo, eliminando così le limitazioni imposte dal browser.

Tuttavia mancava una versione dell'applicazione per i dispositivi iOS. Pertanto si è deciso di implementare una nuova applicazione che, a partire da un unico codice sorgente, potesse essere distribuita per entrambi i sistemi operativi.

